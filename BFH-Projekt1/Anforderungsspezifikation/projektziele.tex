\chapter{Projektziele}

\section{Stakeholder}
\begin{itemize}
\item Benutzer
\item Administrator
\item Auftraggeber UPD
\item Auftragnehmer
\item Humanforschungsverordnung, HFV
\item Humanforschungsgesetz, HFG
\item BFH-Betreuer
\end{itemize}

\subsection{Benutzer}
Das System erleichtert die Zusammenarbeit und übernimmt die Aufgabe automatisch ein Backup zu erstellen. Es läuft im Hintergrund und die konfiguration/installation soll einfach sein. Automatisierter Start des Systems, Einstellungen sind simpel oder vom Administrator verwaltet.

\subsection{Administrator}
Hat ein Handbuch wie er vorgehen muss. Bei korrekter Einrichtung soll er keine weiteren Aufgaben mehr haben, das System soll ohne regelmässige Überprüfung wie vorgesehen funktionieren.

\subsection{Auftraggeber UPD}
System ersetzt vorhandenes, nicht ideales System. Die Daten werden aus der Cloud geholt und nur noch lokal abgespeichert. Kosten werden eingespart, Daten sind schneller verfügbar und das Humanforschungsgesetz wird eingehalten. Die Sicherheitsanforderungen sind erfüllt (Sichere Übertragung der Daten).

\subsection{Auftragnehmer}
Sammelt Erfahrungen bei einem grösseren Entwicklungsprojekt. Erschafft ein System für den Auftraggeber, dass produktiv genutzt werden kann. Arbeitet sich in eine für ihn neue Programmiersprache GO ein und lernt deren Syntax und Semantik.
Der BFH-Betreuer ist zufrieden und befindet alle Anforderungen als erfüllt.

\subsection{Humanforschungsgesetz}
Alle Anforderungen des Humanforschungsgesetzes werden erfüllt.

\subsection{Humanforschungsverordnung}
Die Anforderungen des Humanforschungsgesetzes werden hier detailliert beschrieben, deshalb wird die Verordnung als Grundlage der Anforderungen an das System genutzt.

\subsection{BFH-Betreuer}
Alle Termine werden eingehalten und Know-how wird aufgebaut, parallel im Projektmanagement sowie im Projekt 1. Das System entspricht den Anforderungen und die Dokumentation ist vollständig.

