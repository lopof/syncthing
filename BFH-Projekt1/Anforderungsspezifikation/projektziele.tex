\chapter{Projektziele}

\section{Stakeholder}

\begin{tabular}{ll}
 \parbox{5cm}{
 \begin{itemize}
  \item Benutzer
  \item Administrator
  \item Auftraggeber UPD
 \end{itemize}}
 &
 \parbox{5cm}{
 \begin{itemize}
  \item Auftragnehmer
  \item Gesetzgeber
  \item BFH-Betreuer
 \end{itemize}}

\end{tabular}

\subsection{Benutzer}
Benutzer sind Forscher die an einem Projekt mitarbeiten, dass von der Ethikkommission abgesegnet worden ist. Das System synchronisiert die Daten innerhalb eines Projekts zwischen den Benutzer und dem Administrator. Manuelle Backups der Benutzer werden überflüssig.

\subsection{Auftraggeber UPD}
Das Humanforschunsggesetz wird eingehalten beim Umgang mit Forschungsdaten. Die Sicherheitsanforderungen sind erfüllt (Sichere Übertragung der Daten). Der Schritt von den Rohdaten bishin zu den bewegungskorrigierten Bildern soll automatisiert werden. Ein Bericht der Korrektur soll als PDF ausgegeben werden.

\subsection{Auftragnehmer}
Sammelt Erfahrungen bei einem grösseren Entwicklungsprojekt. Erschafft ein System für den Auftraggeber, dass produktiv genutzt werden kann. Arbeitet sich in eine für ihn neue Programmiersprache GO ein und lernt deren Syntax und Semantik.
Der BFH-Betreuer ist zufrieden und befindet alle Anforderungen als erfüllt.

\subsection{Gesetzgeber}
Alle Anforderungen des Humanforschungsgesetzes werden erfüllt.

\subsection{BFH-Betreuer}
Alle Termine werden eingehalten und Know-how wird aufgebaut, parallel im Projektmanagement sowie im Projekt 1. Das System entspricht den Anforderungen und die Dokumentation ist vollständig.
